\chapter{3.Histogramm and first applicatsion}
\section{The histogramm}
%
% Definition histogramm
%
\begin{definition}[histogram]
	Let $\Omega \subset \Z^d$, $F\subset \R$ discrete and 
 	$u: \Omega \to F$ a discrete discrete image. 
 	The function 
 		$$ H_u : F \to \N_0 \; (:= \N \cup \set{0})$$
 	with 
	$$ H_u (k) := \# \set{ x\in \Omega: 
		u(x) = k}, \quad k\in F$$
	is called \emph{histogramm} of the image $u$.
\end{definition}

%
% Bemerkung zum Histrogramm
%
$H_u(k)$ counts how often colour $k$ appears in $u$.
%
	$$ \sum_{k\in F} H_u (k) = |\Omega| 
		= \hbox{number of pixels in the whole  image}$$
or
%
	$$ {H_u(k) \over |\Omega|} = 
		\hbox{relative frequence$%
			_{\lower0.84em\hbox{\hskip-8.1em (relative Häufigkeit)}}$ 
				 \hskip-.8em \hbox{of colour $k$ in image $u$}} $$

%
% Beispiel  
%
%%%%%%%%%%%%%%%%%%%%%%%%%%%%%%
%%%% minipages mit piks
\newcommand{\pikA}{%
	\begin{minipage}{0.14\linewidth}
		\tikzpictureQNINEONE
	\end{minipage}%
}
\newcommand{\pikB}{%
	\begin{minipage}{0.35\linewidth}
		\tikzpictureQNINETWO
	\end{minipage}%
}
%%%%%%%%%%%%%%%%%%%%%%%%%%%%%%

\begin{bsp}
	$u =$ \pikA has $H_n =$ \pikB  
\end{bsp}

If $u$ ist a continous image, $H_u$ can be understood as
measure (generalized function)\footnote{density
of a probability distribution}. 

%%%%%%%%%%%%%%%%%%%%%%%%%%%%%%
%%%% Pfeil Befehl
\newcommand{\underarrow}[2]{%
	\newlength{\widthA}
	\newlength{\widthB}
	\settowidth{\widthA}{#1}
	\settowidth{\widthB}{#2}
	%\hbox{\vbox{#1$\uparrow$#2}}%
	\vbox{1\widthA}{#1\\ \hskip-2em\parbox{\widthB}{$\uparrow$\\#2}}
	%_{\lower1em\hbox{$\uparrow$ \\ #2}}
}
%%%%%%%%%%%%%%%%%%%%%%%%%%%%%%
{\small 
	Another way to writ this:
	\[ H_u(k) = \sum_{x\in \Omega} 
	\delta _{u(x)}(k),\;
		k \in F \qquad \qquad
		H_u (k) = \int_{\Omega} \delta_{u(x)}(k) dx, \;
		k\in F\]
}

%Linabilder 20171023_1256_53

