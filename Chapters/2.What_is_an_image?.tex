\chapter{2.What is an image?}
\section{Discrete and continuous images}

There are (at least) two different points of view:\\

%%%%%%%%%%%%%%%%%%%%%%%%%%%
\def\lenA{0.1\linewidth}
\def\lenB{0.45\linewidth}
\def\lenC{0.45\linewidth}
%%%%%%%%%%%%%%%%%%%%%%%%%%%

\begin{minipage}[t]{\lenB}
	\ContinousImage
\end{minipage}
\hfill\vrule\hfill
\begin{minipage}[t]{\lenC}
	\DiscreteImage
\end{minipage}

\ \\

\begin{tabular}{p{\lenA} p{\lenB} p{\lenC}}
	\textbf{object:} & matrix 															  & function \\
	\textbf{tools:}  & linear algebra (SVD, \dots) 					  & analysis (differentrage, integrate, 
		\dots) \\
	\textbf{pros:}   & (finite storage) storage, complexity   & freedom, tools,
		\todomp{motions?P.4} \par(e.g. edge discontinuity)\\
	\textbf{cons:}   & limitations: zooming, rotations, \dots & storage (infinite amout of data)\\
\end{tabular}

~\par
arguably, one has:
%
	\begin{itemize}
	  \item real life $\df$ continuous \enquote{images} (objects) 
		\item digital camers $\df$ discrete images
	\end{itemize}	
%
In general we will say:
%
\begin{definition}[(mathematical) image]
	A (mathematical) \emph{image} is a function	
		$$ u: \Omega \to F,$$
	\begin{tabbing}
	where: 
		\= $\Omega \subset \Z^d$ (discrete) or $\Omega\subset \R^d$ (continuous) 
			\dots \emph{domain}\\
		\> $d = 2$ (typical case 2D), $d=3$ (\enquote{3D image} = body or 
	 		$\underbrace{\text{2D + time}}_{\text{movie}}$)\\
		\> $d = 4$ (3D + time) \par
	\end{tabbing}
	
	\begin{tabbing}
		$F$ \dots \= \emph{range of colours}\\
			\> $F= \R$ or $[0,\infty]$ or $[0,1]$  or $\set{0,\dots 255}$, \dots
				grayscale (light intensity) \\
			\> $F\subset \R^3$ \dots RGB image (colored)\\
			\> $F = \set{0,1}$ \dots black/white \hspace{2em}
				\begin{minipage}[c]{0.4\linewidth}
					\tikzpictureFourThree[scale=0.5]
				\end{minipage}
				\begin{minipage}[c]{0.3\linewidth}
					3 Layers\\
					$\df$ colored images:w
				\end{minipage}
	\end{tabbing}

\end{definition}

\todoKom{Matlab stuff}

Large parts of the course: analytical approach (i.e. continuous domain $\Omega$)\\
Since we want to differentirate, \dots the image $u$.
\begin{itemize}[]
  \item[Still:] need to assume that also $F$ ist continuous 
		(not as $\set{0,1}$, $\set{0,1,\dots,255}$ or $\N$)\\
		since otherwise the only differentiable (actually, the only continuous)
		functions $u: \Omega \to F$ are \emph{constant} functions 
		$\aq$ single-colour images
	\item[Also:] We usually take $F$ one-dimensional $(F \subset \R)$. 
		Think of it as either
			\begin{itemize}[-]
			  \item gray scaled image, or
				\item treating R,G \& B layer separately
			\end{itemize}
\end{itemize}

\section{Switching between discrete and continuous images}

\textbf{\large continuous $\to$ discrete:}\\
\begin{itemize}
  \item divide the continuous image in small squared pieces (boxes) 
	(superimpose grid) 

		\begin{minipage}[b]{0.6\linewidth}
		\hspace{-1em}$\bullet\;\:$now: represent each box by \emph{one} 
			value
			\begin{enumerate}[- str{a}tegy 1:]
			  \item take function value $u(x_i)$ \\
					\hspace{4em} for $x_i =$ midpoint of box $B_i$ 
				\item use mean value
					$$ \frac{1}{|B_i|}\int_{B_i} u(x) dx$$
			\end{enumerate}
		\end{minipage}%
		\begin{minipage}[t]{0.4\linewidth}
			\raisebox{1em}{\tikzpictureFIVEONE}
		\end{minipage}
\end{itemize}
$\df $ discrete image
\begin{enumerate}[str{a}tegy 1:]
  \item simple (and quick) but problemativ 
		($u(x_i)$ might represent $u|_{B_i}$ badly; 
		for $u\in L^p$, single point evaluation not
		even defined)
	\item more komplex but also more \enquote{democratic} 
		(actually closer to the way how CCD Sensors in 
		digital camers work)
\end{enumerate}
often the image value of the box $B_i$ gets also digitized, i.e.
fitted (by scaling \& rounding) into range $\set{0,1,dots,255}$
~\\
\par
\textbf{\large discrete $\to$ continous }\\

This is of course more tricky \dots

\begin{tabbing}
$\bullet$ Question:  $\quad$\= \kill

$\bullet$ Again: \> each pixel of the discrete image 
	 corresponds to a \enquote{box} of the continuous image \\
	\> (that is still to be constructed) \\

$\bullet$ Usually: \> pixel value $\mapsto$ \= function value at
	the \emph{midpoint} of the box \\
$\bullet$ Question: \> How to get the other function values 
	(in the box)?
\end{tabbing}

\hspace{1em}
\begin{minipage}[t][2cm][t]{0.30\textwidth}
	\tikzpictureSIXONE	 
\end{minipage}%
\begin{minipage}[c][2cm][t]{0.55\textwidth}
	\begin{tabbing}
 		\underline{idea 1:} $\;$ \= just take the function value of the 
			nearest\\ 
			\> midpoint (\enquote{nearest neighbour interpolation}) 
	\end{tabbing}
\end{minipage}
\vspace{-2.5em}
\begin{center}
For each $x\in B_i: u(x) := u(x_j) \;$ 
where $\displaystyle |x-x_j| = \min_{k} |x-x_k|$
\end{center}
\vspace{-.5em}
%
\begin{minipage}{0.3\linewidth}
 \tikzpictureSIXTWO
\end{minipage}
%
\begin{minipage}{0.7\linewidth}
		$\df \quad$ $u(x) = u(x_i)$ for all $x\in B_i$\\
		$\df \quad$ each box is uni-color\\
		$\df \quad$ the continuous image is essentially still discrete
\end{minipage}
~\\
~\\
{\underline{idea 2}: (bi-) lineare interpolation}
%\todo[inline,color=red]{hier fehlt idea 2 bis P.7}

\begin{minipage}{0.3\linewidth}
	\tikzpictureQSIXTHREE
\end{minipage}
\begin{minipage}{0.7\linewidth}
%
Let $a,b,c,d \dots$ function values at $4$ surroundring adjacemt midpointes
($\nearrow$ figure)\\
$\alpha, \beta, 1-\alpha, 1 - \beta \dots$ distance to dotted lines 
($\nearrow$ figure, w.l.o.g, bob is $1 \times 1$)
\end{minipage}

interpolation (linear) on the dotted line between $a$ and $b$:

\begin{minipage}{0.4\linewidth}
	\tikzpictureQSEVENONE

	\tikzpictureQSEVENTWO
\end{minipage}
%
\raisebox{-2.6em}{\vbox{
	$e := a + \alpha (b-a) = (1-\alpha) a + ab \quad $
	\\
	\medskip
	(1D - interpolation, convex combination)
	\\ \medskip
similarly: $\qquad$ $f = (a-\alpha) c + ad$ 
\\
Then: $\quad$ The same 1D-interpolation between $e$ and $f$
}}

\begin{minipage}{0.3\linewidth}
\tikzpictureQSEVENTHREE
\end{minipage}
\raisebox{0em}{
\begin{minipage}{0.7\linewidth}
	$\begin{aligned}
	\df u(x) &:=  (1-\beta) \cdot e  + \beta \cdot f \\
		&= (1-\beta) [(1-\alpha) a + ab] + \beta [(1- \alpha) c + \alpha d] \\	
		&= \underbrace{(1 - \alpha) (1-\beta)} a + \underbrace{\alpha (1 - \beta)} b 
			+ \underbrace{(1-\alpha) \beta} c + \underbrace{\alpha \beta} d
	\end{aligned}$
	\begin{tikzpicture}
		\draw[help lines,white] (0,0)  grid (9,1);
		\draw (2.7,1) to [out=270,in=180] (3.8,0.17);
		\draw (5.2,1) to [out=270,in=90] (5.1,0.4);
		\draw (7.2,1) to [out=270,in=45] (5.8,0.38);
		\draw (8.8,1) to [out=220,in=0] (6.2,0.3);
		\node at (5,0.2) {\small $\in [0,1] \wedge \Sigma =1 $};
	\end{tikzpicture}
\end{minipage}
}
$\df$ \underline{convex combination} of the function values 
$a,b,c,d$ at the the surrounding $4$ midpoints 
(on which points is the nearest instead of taking just $a,b,c$ or $d$ - depending)
\\
$\df$ 2D linear interpolation, \textit{bi-linear interpolation} (can be 
interpreted as spline interpolation with bilinear \todomp{basis} splines). 

\begin{bsp}

Rotate image \ \ \ \
%
\begin{minipage}{0.2\linewidth}
	\tikzpictureQEIGHTONE
\end{minipage}
%
by angle \ $\phi \neq k\cdot \frac{\pi}{2}$

~\\
%%%%%%%%%%%%%%%%%%%%%%%%%%%%%%%
%%%% Vektoren und Matrizen
\newcommand{\x}{\begin{pmatrix} x_1 \cr x_2 \end{pmatrix}}
\newcommand{\y}{\begin{pmatrix} y_1 \cr y_2 \end{pmatrix}}
\newcommand{\D}{
	\begin{pmatrix}  
		\cos \varphi & -\sin \varphi \cr
		\sin \varphi & \cos \varphi
	\end{pmatrix}%
	_{\lower1.1em\hbox{\hskip-8em 2D rotation matrix}}
}
%%%%%%%%%%%%%%%%%%%%%%%%%%%%%%%
\begin{itemize}
  \item continuous image case: no problem \\
		\begin{minipage}{1\linewidth}
			\tikzpictureQEIGHTTWO	
		\end{minipage}
	$$ \begin{aligned}
	x 	&= D_\varphi \; y \qquad x = \x ,\; y = \y , \; 
		 D_\varphi = \D \cr
	y 	&= \inv{D_\varphi} \; x = D_{-\varphi} \; x \cr
	\end{aligned} $$
	%
	$ \df v(x) := u(y) = u(D_{-\varphi} \; x) \quad
			\forall x\in \hbox{domain of the rotated image}
				%$^{\raise0.5em\hbox{\hskip-6.1em domain of the}}$}
	$ 
\end{itemize}

\begin{itemize}
  \item discrete image case: problem ! 
		\begin{itemize}[]
		  \item For $x\in$ domain of notated image,
						in general {$D_{-\varphi} \; x \not\in $
						domain of original image\footnote{it's not an integer}}\\
						Way out: $ v(x) := $ \emph{interpolation} between the
						$u(\cdot)$ of the 4 surrounding pixels of $D_{-\varphi}$ 
		\end{itemize}
	\begin{minipage}{1\linewidth}
		\tikzpictureQEIGHTTHREE	
	\end{minipage}
\end{itemize}

Something to think about:
	\vspace{-1em}
  \begin{itemize}[]
    \item% 
			What happens in the limit (?) if we, starting
			with an image (discrete or continuous),
			repeatedly switch between discrete and continuous,
			non-stop \dots ? \\
			Does the answer depend on the way of switching ? 
			(continuous $\to$ discrete: midpoint or average, 
			discrete $\to$ continuous: nearest neighbour or bilinear?) 
  \end{itemize}

\end{bsp}


