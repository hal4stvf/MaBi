\documentclass[
 %ngerman,convert={density=300,size=1080x800,outext=.png}]{standalone}
 ngerman]{report}
\usepackage{mathmacros}
\usepackage{exercise}

\begin{document}
\section*{Blatt 3. Aufgabe 2}

\begin{enumerate}[1.]
	\item $f \in L^1(\R^d)$, dann ist $\hat{f}$ beschränkt und stetig. 
%% % % % % % % % % % % % % % % % % % % % % % % % % % % % % % % % % % % % % % %  
	\newcommand{\g}{\underbrace{|e^{i \fop{z,x }}}_{= 1}|}
	\newcommand{\vorfaktor}{C} % hier noch Vorfaktor eintragen
%% % % % % % % % % % % % % % % % % % % % % % % % % % % % % % % % % % % % % % %  
	\begin{proof}
		Für die Beschränkheit gilt es zu zeigen:
		$$ \sup_{z\in \R^d} |\hat{f}(z)| < \infty $$
		Schreibt man dies aus sieht man:
		$$ \begin{aligned}
			|\hat{f}(z)| 
			&= \vorfaktor |\int_{\R^d} f(x) e^{i \fop{z,x }} dx|
			\\
			&\leq \vorfaktor \int_{\R^d} |f(x)| \g dx %für g siehe oben
			\\
			&= \vorfaktor \int_{\R^d} |f(x)| dx = \vorfaktor \norm{f}_1 < \infty 
		\end{aligned} $$

		Nun zur Stetigkeit. Es muss gezeigt werden, dass für $(z_n)_{n\in\N}\in \R^d$
		mit $\limes z_n = z$ gilt
		
		$$ \begin{aligned}
			\limes \hat{f}(z_n) &= \hat{f}(z)
			\\
			\limes \vorfaktor \int_{\R^d} f(x) e^{i \fop{z_n,x }} dx 
		&= \vorfaktor \int_{\R^d} f(x) e^{i \fop{z,x }} dx 
		\end{aligned}$$	
		Dies ist gerade die Aussage des \enquote{Satz von Lesbegue} 
		(wenn auch etwas versteckt):

		{\scriptsize
			Es seien $\phi_n,\phi: S \to Y$ $\mu$-messbar und $g \in \text{L}(\mu;\R)$.
			($S$ beliebige Menge, $Y$ Banachraum)

			Gilt 
			\begin{enumerate}[(1)]
				\item $\phi_n \to \phi$ $\mu$-fast überall $\quad n \to \infty$
			  \item $|\phi_n| \leq g$ $\mu$-fast überall $\quad \forall n\in \N$ 
			\end{enumerate}		

			Dann sind $\phi_n,\phi \in \text{L}(\mu, \R)$ und es konvergiert 
			$$\limes \int |\phi_n| d\mu = \int |\phi| d\mu$$
		}
		
		\textbf{Vorbereitung:}	
		
		Definiere: 
		$\phi: \R^d \times \R^d \to \R$, $ (x,z) \mapsto \vorfaktor 
		f(x)e^{-i \fop{z,x }} $
		
		Wir wollen dann soetwas haben:	

		$$\limes \int |\phi_n(x)| d\mu(x) = \int |\phi(x)| d\mu(x)$$

		Damit definieren wir: $\phi_n(x) := \phi_{z_n}(x) := \phi(x,z_n)$ 
		mit $(z_n)_{n\in \N} \in \R^d$

		Dann brauchen wir um den obrigen Satz anwenden zu können:
		
		\begin{enumerate}[(1)]
		  \item $\phi_n,\phi_z $ 
			\item Stetigkeit von $\phi_z(x)$  
		\end{enumerate}
	\end{proof}
	%%% %%% %%% %%%%%% %%%%%% %%%%%% %%%%%% %%%%%% %%%%%% %%%%%% %%%%%% %%%%%% %%%
	\item $c\in \R \setminus \set{0}$, $f\in L^1(\R^d)$. Es sei $g(x) := f(cx)$. 
		Es gilt: $\hat{g}(z) = \frac {1} {|c|^d} \hat{f} ( \frac {z} {c})$.
	%%% %%% %%% %%%%%% %%%%%% %%%%%% %%%%%% %%%%%% %%%%%% %%%%%% %%%%%% %%%%%% %%%
	\item $f \in L^1(\R^d)$. Es sei $g(x) := \overline{f(x)}$.
		Es gilt: $\hat{g}(z) = \overline{\hat{f}(-z)}$.
	%%% %%% %%% %%%%%% %%%%%% %%%%%% %%%%%% %%%%%% %%%%%% %%%%%% %%%%%% %%%%%% %%%
	\item Es sei $f\in L^1(\R^d)$ und $g(x) := f(x+a)$ für ein $a\in \R^d$.	
		Dann gilt $\hat{g}(z) = e^{i\langle a, z \rangle}\hat{f}(z)$.
\end{enumerate}
\section*{Blatt 3. Aufgabe 3. Gaußkern}
Nur eindimensionalen Fall betrachet ...
$G_{\sqrt{2t}} = \frac {1} {(4 \pi t)^{d/2}} 
\cdot e^{ \frac {-|x|^2} {4t}} $

\newcommand{\Dt}{\frac {\partial} {\partial t}}
$$\Dt G_{\sqrt{2t}} (x) = \Delta G_{\sqrt{2t}} (x) $$
\begin{minipage}{0.5\linewidth}
  \small
	$ \begin{aligned}
		\Dt G_{\sqrt{2t}} (x) 
		&=  \Dt \frac {1} {(2\pi\cdot 2t)^{ \frac {d} {2}}}\cdot 
		exp({ \frac {-|x|^2} {4t}})
		\\
		&= \left(\Dt \frac {1} {(4 \pi t)^{ \frac {d} {2}}}\right)
		e^{ -\frac {|x|^2} {4t}} + \frac {1} {(4 \pi t)^{ \frac {d} {2}}}\cdot  
		\left(\Dt e^{ -\frac {|x|^2} {4t}}\right)
		\\
		&=  \frac {- \frac {d} {2}4\pi } {(4\pi t)^{ \frac {d} {2} - 1}} \cdot 
		e^{ \frac {-|x|^2} {4t}}
		+ \frac {1} {(4 \pi t)^{ \frac {d} {2}}}\cdot  
		\left(\Dt \frac {-|x|^2} {4t}\right)e^{ \frac {-|x|^2} {4t}}
		\\
		&=  \frac {-2d\pi } {(4\pi t)^{ \frac {d} {2} - 1}} \cdot 
		e^{ \frac {-|x|^2} {4t}} + \frac {1} {(4 \pi t)^{d/2}}\cdot  
		\left(\frac {|x|^2} {4t^2}\right)e^{ \frac {-|x|^2} {4t}}
		\\
		&=  \left(\frac {-2d\pi } {(4\pi t)^{ \frac {d-2} {2}}} 
		+ \frac {1} {(4 \pi t)^{d/2}}\cdot  
		\frac {|x|^2} {4t^2} \right ) \cdot e^{ \frac {-|x|^2} {4t}}
		\\
		&=  \left(\frac {-2d \pi } {\sqrt{(4\pi t)^2}} \cdot 
		\frac {1} {(4 \pi t)^{d/2}} + \frac {1} {(4 \pi t)^{d/2}}\cdot  
		\frac {|x|^2} {4t^2} \right ) \cdot e^{ \frac {-|x|^2} {4t}}
		\\
		&=  \left(\frac {-d } {2t} + \frac {|x|^2} {4t^2} \right )
		\frac {1} {(4 \pi t)^{d/2}}\cdot e^{ \frac {-|x|^2} {4t}}
		\\
	\end{aligned} $
\end{minipage}
\small
\begin{minipage}{0.5\linewidth}
  $$ \begin{aligned}
		\Delta G_{\sqrt{2t}} (x) 
		&=\Delta \frac {1} {(4 \pi t)^{d/2}} 
		\cdot e^{ \frac {-|x|^2} {4t}}
		\\
		&= \frac {1} {(4 \pi t)^{d/2}} \Delta  e^{ \frac {-|x|^2} {4t}}
		\\
		&= \frac {1} {(4 \pi t)^{d/2}} \sum_{i\in\set{1,..,d}} 
		\partial_i\left(
		(\partial_i \frac {-|x|^2} {4t} )e^{ \frac {-|x|^2} {4t}}
		\right)
		\\
		&= \frac {1} {(4 \pi t)^{d/2}} \sum_{i\in\set{1,..,d}} 
		\partial_i\left(
		(\frac {-2x_i} {4t} )e^{ \frac {-|x|^2} {4t}}
		\right)
		\\
		&= \frac {1} {(4 \pi t)^{d/2}} \sum_{i\in\set{1,..,d}} 
		\left(
		(\partial_i \frac {-2x_i} {4t} )e^{ \frac {-|x|^2} {4t}}
		+ (\frac {-x_i} {2t} )\partial_ie^{ \frac {-|x|^2} {4t}}
		\right)
		\\
		&= \frac {1} {(4 \pi t)^{d/2}} \sum_{i\in\set{1,..,d}} 
		\left(
		(\frac {-1} {2t} )e^{ \frac {-|x|^2} {4t}}
		+ (\frac {-x_i} {2t} )
		(\frac {-x_i} {2t} )e^{ \frac {-|x|^2} {4t}}
		\right)
		\\
		&= \frac {1} {(4 \pi t)^{d/2}} e^{ \frac {-|x|^2} {4t}}
		\sum_{i\in\set{1,..,d}} 
		\left( \frac {-1} {2t} + \frac {x_i^2} {(2t)^2}
		\right) 
		\\
		& = \frac {1} {(4 \pi t)^{d/2}} e^{ \frac {-|x|^2} {4t}}
		\left(\left(\sum_{i\in\set{1,..,d}} 
		\frac {-1} {2t}\right) + \frac {|x|^2} {(2t)^2}
		\right) 
		\\
		& = \frac {1} {(4 \pi t)^{d/2}} e^{ \frac {-|x|^2} {4t}}
		\left(
		\frac {-d} {2t} + \frac {|x|^2} {(2t)^2}
		\right) 
	\end{aligned}$$
\end{minipage}

\textbf{Für Interessiert}

Es sei $f \in C_b(\R^d)$. 

	$$ \forall x \in \R^d : \lim_{t\searrow 0} (G_{\sqrt{2t}} \ast f)(x) = f(x) $$

$$\lim_{t\searrow 0} \int_{\R^d} \frac {1} {\sqrt{4t}} e^{ \frac {- |z-x|^2} 
	{4t}}f(z) dz = f(x)$$
\end{document}
