\documentclass[
 %ngerman,convert={density=300,size=1080x800,outext=.png}]{standalone}
 ngerman]{report}
\usepackage{mathmacros}
\usepackage{exercise}

\begin{document}
\section*{Blatt 2. Aufgabe 3}

\begin{enumerate}[1)]
	\item $(f \ast g) \ast h = f \ast (g \ast h)$ 
		\vspace{-1mm}
		\begin{proof}%
			\[\begin{aligned}
				(f \ast g) \ast h (k) = \sum_{j\in\Z} 
					\left(f \ast g)(	k - j ) h(j) \right)
				&	= \sum_{j\in\Z} \left(\sum_{l \in \Z} f(k-(j+l)\right) \ast g(l)) h(j)
			  \\
				& = \sum_{\xi\in\Z} \sum_{\kappa \in \Z} 
					f(k-\xi) g(\xi - \kappa) h(\kappa) 
				\\
				& = \sum_{\xi\in\Z} f(k-\xi) \left(\sum_{\kappa \in \Z} 
					g(\xi - \kappa) h(\kappa) \right)
				\\
				 = \sum_{j\in\Z} f(k-j) (g \ast h) (j) 
				 & = f \ast (g \ast h) (k)
			\end{aligned}\]
		\end{proof}
	\item $f\ast g = g \ast f$. 
	\item $f \in \ell^(\Z), g \in \ell^q(\Z)$. Dann ist $f\ast g \in \ell^r (Z)$ 
		mit $ \frac {1} {p} + \frac {1} {q} = 1 + \frac {1} {r}$, $p,q,r \geq 1$
		und 
		$$ \norm{f\ast g}_{L^r} \leq \norm{f}_{L^p} \cdot  \norm{g}_{L^q}$$
	\item $\overset{\sim}{f} \ast \overset{\sim}{g} = \overset{\sim}{f\ast g}$
\end{enumerate}
\end{document}
